\documentclass[a4paper,11pt]{article}
\usepackage[utf8]{inputenc}
\usepackage{helvet}
\usepackage{graphicx}
\usepackage{color}
\usepackage{geometry}
\usepackage[T1]{fontenc}
\usepackage[unicode]{hyperref}
\usepackage{amsmath}
\usepackage{gensymb}
\usepackage{multirow}
\geometry{hmargin={2cm, 2cm}, height=10.0in}
\usepackage{multicol}


\usepackage{lipsum} 
\usepackage{indentfirst}

\usepackage{listings}
\usepackage{color}

\definecolor{dkgreen}{rgb}{0,0.6,0}
\definecolor{gray}{rgb}{0.5,0.5,0.5}
\definecolor{mauve}{rgb}{0.58,0,0.82}

\lstset{frame=tb,
  language=Java,
  aboveskip=3mm,
  belowskip=3mm,
  showstringspaces=false,
  columns=flexible,
  basicstyle={\small\ttfamily},
  numbers=none,
  numberstyle=\tiny\color{gray},
  keywordstyle=\color{blue},
  commentstyle=\color{dkgreen},
  stringstyle=\color{mauve},
  breaklines=true,
  breakatwhitespace=true,
  tabsize=3,
  basicstyle=\tiny
}

\title{Title}
\author{Wojciech Surówka}

\begin{document}
\begin{enumerate}
 \item Wstęp:
 \begin{enumerate}
   \item Słowniczek 
   \item Cel pracy 
   \item Wstęp Teoretyczny \\ Najpierw napiszę dokumentację potem będę opisywać definicje i wzory które były mi potrzebne do opisu. Ale przyjmę sugestię. 
 \end{enumerate}
 \item Projekt: tu jakiś wstęp + opis otrzymywanych sygnałów z układu RXHDR
 \begin{enumerate}
   \item Rozwiązania z przerwaniami przy pomocy języka Arduino
   \item Rozwiązanie z przerwaniami przy pomocy bezpośredniego programowania procesora ARM32 AT91SAM3X8E \\ krótki opis mojej programistycznej porażki
   \item Rozwiązanie z wsparciem zewnętrznego hardweare:
   \begin{enumerate}
     \item Opis płytki wraz z korzystnymi wyjściami i schematami blokowymi.
     \item Opis i dokumentacja oprogramowania arduino due. Duże skupienie na części o istotnym timing'u. 
     \item Opis i dokumentacja oprogramowania znajdującego się po stronie PC. Zdjęcia GUI. 
   \end{enumerate} 
 \end{enumerate}
  \item Wyniki badań:
  \begin{enumerate}
    \item Wyniki dla przerwań, opis dlaczego to rozwiązanie się nie nadaje. 
    \item Wyniki dla rozwiązania końcowego, te pierwsze: dzięki nim wyjaśnienie potrzeby używania stałych regulujących prace (uwzględnienie czasu martwego itd)
    \item Jestem optymistą i wierzę że pod koniec semestru będę mógł w panice dorobić parę końcowych doświadczeń i tak na 100\% dokręcić stałe. 
  \end{enumerate} 
  \item Podsumowanie:

  \item bibliografia.
\end{enumerate}
\end{document}

\section{Wyniki badań}

Poniżej umieszczone są wyniki badań z zastosowaniem użytych rozwiązań. 

\subsection{Rozwiązanie programowe na platformie Arduino}

Przeprowadzono testy polegające na bezpośrednim zliczaniu sygnałów generowanych przez generator zewnętrzy. 
Impulsy napięciowe z generatora kształtem oraz długością odpowiadały tym produkowanym przez układ RXHDR\_V1 \cite{master}.

Przeprowadzono po pięć badań na każdą częstotliwość i wyliczono błąd względem spodziewanego wyniku. 
Czas akwizycji ustalono na jedną sekundę sprawiając że liczba spodziewanych zliczeń jest równa częstotliwości generowanych sygnałów.

Wyniki znajdujące się w tabeli \ref{rts table} (Aneks) zwizualizowane są na wykresie \ref{rts wyniki}.



Wyniki pokazują dokładność mieszczącą się w  1\textperthousand, 
jednak błąd w otrzymanym wyniku zależy od częstotliwości generowanych zliczeń.
Może to być spowodowane interferencją przerwania powodującego zliczenie z innymi przerwaniami wymaganymi do działania mikrokontrolera.

Po osiągnięciu częstotliwości granicznej > 0.715 [MHz] program mikrokontrolera przestaje wysyłać dane na komputer zewnętrzny.
Jest to spowodowane tym że zaraz po wyjściu z obsługi przerwania systemowego program natychmiast zaczyna obsługę następnego przerwania. 
Liczba graniczna pozwala przybliżyć czas konieczny na wykonanie jednego przerwania na podstawie przekształcenia wzoru \ref{Cykli w sec}. 
$$ t_p = \frac{1}{f_p} = \sim 1.3986 [\mu s] $$
$$ N_c = \frac{t_p}{t_c} =  \frac{1.3986 \mu s}{11.9 ns} =\sim 118$$
gdzie: \\
        \indent $t_p$ -  czas potrzebny na obsługą przerwania\\
        \indent $t_c$ -  czas jednego cyklu procesora (dział \ref{dzial arduino} ) \\
        \indent $f_p$ -  częstotliwość graniczna przerwań \\
        \indent $N_c$ -  ilość cykli koniecznych na pojedyncze przerwanie \\

Liczba cykli na przerwanie jest mniejsza niż ta szacowana w dziale \ref{dzial arduino} (355 + 128)\cite{ard_opt_git}, wynika to z faktu że poprzednia estymacja była wykonana dla najgorszego przypadku dla nieoptymalizowanego kodu.
Mimo lepszych wyników doświadczalnych niż te szacowane teoretycznie otrzymane wartości nadal odbiegają od optymalnego czasu wywołania przerwania (12 + 10) \cite{interupt latency} i wciąż są znacznie poniżej wymagań projektu. 

Dodatkowe testy potwierdziły że wraz ze zwiększeniem ilości badanych kanałów częstotliwość graniczna zmniejsza się jak $\frac{1}{n}$, gdy $n$ to ilość badanych jednocześnie kanałów. 

\begin{figure}[h]
        \centering
        \includegraphics[width=\textwidth]{rts.jpg}
        \caption{Wyniki testów z wykorzystaniem przerwań systemowych}
        \label{rts wyniki}
\end{figure}

\subsection{Rozwiązanie z układem zewnętrznych liczników buforujących}

Dane do poniższej interpretacji i wizualizacji zostały uzyskane przez analizę plików archiwalnych otrzymywanych przy działaniu programu do kontroli układu.
Pliki te są przechowywane w formacie JSON.

Obserwacja sygnałów na została dokonana na oscyloskopie Keysight 3024A. Generator sygnałów używany w układzie badawczym to generator Tektonix AFG3102.

Sygnał impulsów odpowiadających tym pochodzącym z układu RXHDR\_v1 wytwarzanych przez generator miał następujące parametry: poziom napięcia 1.8V, kształt prostokątny, impuls o szerokości 50 ns.

\subsubsection{Test układu liczników przy użyciu oscyloskopu}

Przed przystąpieniem do badania pełnego układu (RXHDR, układ liczników, mikrokontroler, komputer kontrolujący) przeprowadzono test elektroniki z minimalnym wpływem realizowanego kodu programu na wynik. 

Układ badawczy składał się z układu liczników  otrzymującego sygnały z generatora, charakterystyką odpowiadającym tym z układu RXHDR, oscyloskopu oraz Arduino Duo nadający sygnały zmiany licznika (\textit{read\_CLK}) oraz wyboru multiplexera (rysunek \ref{wybor schema} i \ref{licznik}). Generowany sygnał był dostarczany na pojedyńczy kanał układu liczników. 
Oscyloskop analizował wyjście z generatora oraz sygnał zmiany licznika, dodatkowo za pomocą cyfrowego analizatora poziomów wizualizował wyjścia DOC\_0-3 (Stan wybranego licznika).

Powyższy układ badawczy pozwalał na potwierdzenie funkcjonalności układu elektronicznego oraz:
\begin{itemize}
        \item Wyznaczyć czas cyklu blokowania pojedyńczego licznika równy $\sim 0.3\mu s$.
        \item Potwierdzić istnienie zjawiska:
        \begin{itemize}
                \item braku zliczenia impulsu podczas odczytu licznika,
                \item przepełnienia licznika,
                \item synchronizacji płytki z mikrokontrolerem (powtórzenie stanu licznika co 9 cykli), 
                \item utrzymania wartości licznika między odczytami dla sytuacji gdy sygnał z generatora nie wystąpił między wyborem licznika.
        \end{itemize}  
        \item Porównać czas blokowania licznika dla różnych kanałów. Krok ten pozwolił na odkrycie różnicy w czasie blokowania licznika dla kanałów 8 i 16. W ten sposób dopasowano ilość pustych funkcji koniecznych do wyrównania czasu martwego dla każdego z kanałów (dział~\ref{oprogramowanie mikrokontrolera}).
\end{itemize} 

Przykładowe rzuty ekranu oscyloskopu z tego testu znajdują się na rysunku \ref{Oscyloskop}. Na tych wizualizacjach wartości D0-3 odpowiadają kolejno wartościom linii magistrali danych DOC\_0-3. Oznacza to że wartości logiczne dla tych magistrali prezentują liczbę binarną będącą wartością na liczniku. Sygnał koloru zielonego jest sygnałem zmiany licznika (\textit{readCLC}), rosnące zbocze tego impulsu powoduje przełączenie na odczyt kolejnego licznika. Ponieważ tylko jeden z liczników otrzymuje sygnał tylko co 9 impuls (8 liczników i jeden cykl pusty) powoduje sygnał na liniach DOC\_0-3.

\begin{figure}
        \includegraphics[width=\textwidth]{scope_1.png}
        \includegraphics[width=\textwidth]{scope_2.png}
        \caption{Rzuty ekranu z oscyloskopu podłączonego do płytki. 
        Impulsy wejściowe (żółte) są wprowadzane na jeden z kanałów analogowych, sygnał zmiany wyboru licznika (\textit{readCLC}) jest widoczny na drugim z kanałów(zielony). 
        Sygnał D0-3 pokazuje binarne zliczanie na danym kanale. 
        Na górnym zrzucie widać że w przypadku nie zarejestrowania następnego sygnału wyjście dla kanału się powtarza. 
        Na dolnym obrazku widoczny jest impuls trafiający na czas martwy będący chwilą odczytu wartości z kanału.
        Następuje przeskok z wartości $1101_{2}$ ($13_{10}$) na $0001_{2}$ ($1_{10}$) mimo faktu że wystąpiło 5 impulsów. Oczekiwana wartość licznika to $0010_2$ ($2_{10}$)
        }
        \label{Oscyloskop}
\end{figure}

\subsubsection{Kalibracja układu}
\label{section kaliblracja}

Przed przystąpieniem do końcowych testów układu konieczne jest przeprowadzenie kalibracji ze względu na czas martwy układu. 
Oczekiwanym efektem jest regulacja czasu akwizycji w taki sposób że liczba zliczeń będzie odpowiadać ilości impulsów która powinna być uzyskana w czasie akwizycji przez układ bez czasu martwego. 

Zbadano okres jednego cyklu pomiaru (8 kanałów + cykl pusty) i uzyskano czas cyklu równy $4.52 \mu s$. Taki cykl jest widoczny na rysunku \ref{Oscyloskop} (dolna wizualizacja).
Na podstawie tej wartości można wyliczyć ilość cykli akwizycji przypadających na pojedyńcza milisekundę. Obliczenie ilości cykli przypadających na 1ms konieczne jest do ustalenia wartości zmiennej\textit{CIRCLES\_FOR\_1MS} (kod \ref{code aqw} linia 1) która służy do regulowania ilości przebiegów pętli w określonym czasie akwizycji. Obliczenia konieczne do wyznaczenia tej zmiennej umieszczone sa poniżej:

\begin{equation}
        \label{per cykl}
        CIRCLES\_FOR\_1MS = \frac{1*10^{-3}}{4.52 * 10^{-6}} = 221.238938053
\end{equation}

%Korzystając z tej wartości i wartości współczynnika korekcji czasu martwego równego 1.0 przeprowadzono badania zależności częstotliwości podawanej z generatora do częstotliwości zliczeń dla różnych czasów akwizycji.  

Wartość ta została uzyskana na podstawie wizualizacji procesu akwizycji danych na oscyloskopie. Dodatkową zmienną wpływającą ilość pętli występujących w czasie akwizycji jest współczynnik korelacji czasu martwego. Nazwą tej zmiennej w kodzie programu (\ref{code aqw}) to \textit{DEAD\_TIME\_CORECTION}. Ustalenie wartości tej zmiennej dokonano poprzez analizę zależności częstotliwości podawanej z generatora do częstotliwości zliczeń dla różnych czasów akwizycji. A opis tego procesu jest umieszczony poniżej.

Układ badawczy składał się z komputera, Arduino Due, badanego układu elektronicznego, taśmy transmisyjnej, oscyloskopu, generatora impulsów, zasilacza. Na potrzeby tego badania ustalono tymczasową wartość współczynnika korekcji czasu martwego na wartość 1.0.

Podczas pomiarów wykonywano trzykrotnie próby testowe dla wybranych dziesięciu różnych częstotliwości sygnału testowego z generatora, dla 2 czasów akwizycji. Na podstawie tych danych przygotowano krzywe odpowiedzi zaprezentowane na rysunku \ref{wykresy fit calib}. Do punktów pomiarowych dopasowano funkcję liniową: $f(x) = a*x$, wyniki fitowania umieszczone zostały w tabeli \ref{dead time fit}.

Na podstawie danych z tabeli \ref{dead time fit} ustalono wartość stałej DEAD\_TIME\_CORECTION równe:
\begin{equation}
        \label{dead time eq}
        DEAD\_TIME\_CORECTION = \frac{1}{\sum^n_i \frac{a_i}{n}} = 1.13636
\end{equation} 


Zbadano dodatkową maksymalną częstotliwość sygnału z generatora impulsów wejściowych dla której następuje przepełnienie liczników. Jest to wartość $3.72 [MHz]$. Po przekroczeniu tej wartości częstotliwości sygnału z  generatora wartość zliczeń rejestrowanych przez system drastycznie spada. 

\begin{figure}[h]
        \centering
        \begin{multicols}{2}
                \includegraphics[width=0.5\textwidth]{dead1A6.jpg} \par
                \includegraphics[width=0.5\textwidth]{dead2A7.jpg} \par
        \end{multicols}
        \caption{Wybrane wykresy użyte w celu kalibracji układu. Czas akwizycji 1000 ms lewy wykres oraz 500 ms prawy wykres}
        \label{wykresy fit calib}
\end{figure}

\begin{table}[h]
        \caption{Wyniki fitowania konieczne dla ustalenia korekcji czasu martwego}
        \label{dead time fit}
        \centering
        \begin{tabular}{|c|c|c|c|c|}
                \hline
                Nazwa kanału & czas akwizycji [-] & $a$ [ - ]& $\Delta a$ & konwersja na 1s [ - ] \\ \hline
                1A6 & 1000 ms & 0.879395 & $1.8 * 10^{-5}$ & 0.879395 \\ \hline
                2A5 & 500 ms & 0.440366 & $8.1 * 10^{-6}$ & 0.880732 \\ \hline
                2A7 & 500 ms & 0.440352 & $7.9 * 10^{-6}$ & 0.880704 \\ \hline
        \end{tabular}
\end{table}

\paragraph{Błąd przybliżenia do liczby naturalniej.}
Jak widać w fragmencie kodu \ref{code aqw} liczba cykli przeprowadzonych w trakcie pojedyńczej akwizycji musi być liczbą całkowitą. Liczba cykli jest określona wzorem:
\begin{equation}
        N_c [-] = int(A_t*D_t*f_c)
\end{equation}
Gdzie:
\begin{description}
        \item $N_c$ - liczba cykli w procesie akwizycji,
        \item $A_t$ - czas  akwizycji w ms
        \item $D_t$ - współczynnik korekcji czasu martwego 
        \item $f_c$ - ilość cykli w ms
\end{description}

Błąd wprowadzony w wyniku przybliżenia mieści się w przedziale wartości [0,1) pozwala to na wyliczenie maksymalnego względnego błędu uzyskiwanego w wyniku przybliżania:
\begin{equation}
        \Delta N_{c_{max}} [\%] = \frac{1}{A_t*D_t*f_c}  * 100\%
\end{equation} 
Gdzie:
\begin{description}
        \item $\Delta N_{c_{max}}$ - maksymalny względny błąd,
\end{description}

Podstawiając współczynniki uzyskane w dziale \ref{section kaliblracja} możemy uzyskać wartości maksymalnego błędu dla najczęściej używanych czasów akwizycji. Wartości te znajdują się w tabeli \ref{tab przyblizenie niep}. Na podstawie tych wartości można stwierdzić że błąd popełniony podczas przybliżania zmienia się w zależności od czasu akwizycji jest on pomijalnie mały.

\begin{table}
        \centering
        \caption{Wartości błędu wynikające z przybliżenia liczbą całkowitą}
        \label{tab przyblizenie niep}
        \begin{tabular}{|c|c||c|c|}
                \hline
                Czas akwizycji [ms] &   $\Delta N_{c_{max}} [\%]$&Czas akwizycji [ms] &   $\Delta N_{c_{max}} [\%]$ \\ \hline
                50 & 0.0079 & 100 & 0.0039 \\ \hline
                1000 & 0.00039 & 5000 & 0.000079 \\ \hline
        \end{tabular}
\end{table}

\paragraph{Ograniczenie czasu akwizycji}

Konieczne jest rozważenie problemu przetrzymywania danych zliczeń dla kanału w pojedyńczej zmiennej. Teoretyczny czas akwizycji jest nieskończony jednak w pewnej chwili wartość zmiennej przechowującej dane o zliczeniu zostanie przepełniona.

W celu przechowywania zmiennej zawierającej informacje o ilości zliczeń użyty została zmienna typu \textit{unsigned int}. Dla mikrokontrolera AT91SAM3X8E wartość ta mieści się w 32 bitach. Fakt że zmienna ta nie ma znaku sprawia że wartości graniczne tej liczby to [~0, $2^{33}-1$~]. Oznacza to, że dla zbadanej maksymalnej częstotliwości granicznej układu 3.72 [MHz] możemy wyliczyć czas akwizycji dla której nastąpi przepełnienie dla maksymalnej częstotliwości zliczeń. 
\begin{equation}
        A_{t_{max}} = \frac{N_{max}}{f_{max}} = \frac{2^{33}-1}{3.72 * 10^{6}} = \frac{8589934591}{3.72 * 10^{6}} = 2309.12 [s] = ~6.5 [h]
\end{equation}
Gdzie:
\begin{description}
        \item $A_{t_{max}}$ - maksymalny czas akwizycji ze względu na efekt przepełnienia liczby 32 bitowej
        \item $N_{max}$ - maksymalna wartość liczby 32 bitowej
        \item $f_{max}$ - maksymalna częstotliwość impulsów detekcyjnych przypadająca na pojedynczy kanał
\end{description}

Wartość ta jest absurdalnie duża i nieprzydatna ponieważ przez cały czas akwizycji mikrokontroler jest nieresponsywny dlatego też komputer kontrolujący już po 5 s od wysłania zapytania zakłada że na kontrolerze wystąpił błąd i rozpoczyna proces wysyłania zapytań w celu zbadania problemu. Sprawia to że ze względu tą decyzję maksymalna wartość czasu akwizycji równa się 10 s. Jeżeli konieczne jest zwiększenie tej liczby należy zwiększyć wartość zmiennej WAIT\_FOR. Problem ten nie dotyczy funkcjonalności \textit{Single Shoot}.   

\subsubsection{Testowanie układu liczników za pomocą generatora zewnętrzego.}

Przy użyciu współczynników otrzymanych w dziale \ref{section kaliblracja} oraz układu badawczego odpowiadającemu temu z działu \ref{section kaliblracja} przeprowadzono test układu z wykorzystaniem dzielnika pozwalającego na dostarczenie sygnałów impulsów, odpowiadającym tym generowanym przez układ RXHDR\_V1, do wszystkich kanałów jednocześnie.
Badanie przeprowadzono dla 10 częstotliwości od 100Hz do 3.2MHz. Czas akwizycji badania był równy 1s dzięki czemu liczba zliczeń odpowiada częstotliwości sygnału z generatora. Uśrednione wartości wyników dla wybranych kanałów znajdują się w tabeli \ref{tabela wyniki surowe} (Aneks).

\begin{figure}
        \begin{multicols}{2}
            \includegraphics[width=0.5\textwidth]{hist/hist1.jpg} \par    
            \includegraphics[width=0.5\textwidth]{hist/hist2.jpg} \par    
        \end{multicols} \hfill
        \begin{multicols}{2}
            \includegraphics[width=0.5\textwidth]{hist/hist3.jpg} \par
            \includegraphics[width=0.5\textwidth]{hist/hist4.jpg} \par    
        \end{multicols}
        \caption{Histogramy rozrzutu zliczeń dla wybranych kanałów i częstotliwości}
        \label{hist licz}
\end{figure}

W celu otrzymania krzywych zależności częstotliwości od liczby zliczeń (rysunek \ref{multi wyk}) przeprowadzono serię pomiarów dla 10 różnych częstotliwości. Seria wyników dla wybranego kanału dla wybranej częstotliwości wyznacza punkt pomiarowy użyty do dopasowania funkcji liniowej: $f(x) = a * x$. Niepewność tych punktów została wyznaczona jako odchylenie standardowe wyników dla badanej częstotliwości. Wyniki pojedyńczych pomiarów składających się na punkt pomiarowy zostały wizualizowane za pomocą histogramów na rysunku \ref{hist licz}
%Każdy z punktów pomiarowych składał się z serii osobnych akwizycji. Jako wartość punktu pomiarowego ustalono średnią z pomiaru, a jako niepewność pomiaru (zaznaczona na wykresach i uwzględniona w dopasowaniu krzywej) jako odchylenie standardowe średniej. Wartości częstotliwości pojedynczych pomiarów dla wszystkich kanałów dla wybranych częstotliwości znajdują się na histogramach zawartych na rysunkach \ref{hist licz}. 

Wyniki odchylenia standardowego oraz różnicy między podawaną częstotliwością a liczbą zliczeń umieszczone są na rysunku \ref{3d licznik}. Wartości te podawane są jako wartości bezwzględne oraz wartości związane z częstotliwością generowanych sygnałów.

\begin{figure}
        \centering
        \begin{multicols}{2}
                \includegraphics[width=0.5\textwidth]{3d/difLiczik.jpg} \par                
                \includegraphics[width=0.5\textwidth]{3d/niepLiczik.jpg} \par                
        \end{multicols} \hfill
        \begin{multicols}{2}
                \includegraphics[width=0.5\textwidth]{3d/difWzgLiczik.jpg} \par                
                \includegraphics[width=0.5\textwidth]{3d/niepWzgLiczik.jpg} \par                
        \end{multicols}
        \caption{Wartości odchylenia standardowego oraz różnica między zliczeniami podawane w wartościach bezwzględnych oraz jako wartości względne. }
        \label{3d licznik}
\end{figure}

Wybrane wykresy pomiaru zależności częstotliwości od liczby zliczeń umieszczone są na rysunku \ref{multi wyk} a wyniki fitowania znajdują się w tabeli \ref{multi fit} (Aneks). Wyniki pomiarów fitowane były krzywą $f(x) = a*x$. 
Średnia po wszystkich współczynnikach to $0.999997 \pm 2.5*10^{-5}$
odchylenie standardowe wyników jest równe  $3.94 * 10^{-5}$.

\begin{figure}
        \centering
        \begin{multicols}{2}
                \includegraphics[width=0.5\textwidth]{multi1A3.jpg} \par
                \includegraphics[width=0.5\textwidth]{multi1A6.jpg} \par                
        \end{multicols} \hfill
        \begin{multicols}{2}
                \includegraphics[width=0.5\textwidth]{multi2A2.jpg} \par
                \includegraphics[width=0.5\textwidth]{multi2A7.jpg} \par                
        \end{multicols}
        \caption{Wybrane wykresy testów układu licznika wraz z dopasowanymi krzywymi liniowymi}
        \label{multi wyk}
\end{figure}




\subsubsection{Testowanie układu liczników przy użyciu kanału wbudowanego testowego układu RXHDR\_V1}
\label{section RXHDR test}

Układ badawczy składał się z komputera, Arduino Due, badanego układu elektronicznego, taśmy transmisyjnej, oscyloskopu, generatora impulsów, zasilacza, układu RXHDR\_V1, zasilacza wysokiego napięcia. 

Układ RXHDR\_v1 połączony z zestawem liczników buforujących został zasilony napięciem 3.3V oraz 1.8V a detektor został spolaryzowany poprzez zasilacz wysokiego napięcia wartością 300V. Następnie zestaw ten został połączony z Arduino Due i komputerem kontrolującym. 

Generator połączony z wejściem testowym układu RXHDR\_V1 generował sygnał prostokątny o napięciu 0.3V i częstotliwości 100kHz. 

Pomiar polegał na akwizycji danych (1000ms) dla różnych napięć dyskryminatora w celu utworzenia krzywej-S. Krzywe tych pomiarów umieszczone zostały na wykresach \ref{s curve test}. Na podstawie tych wykresów przeprowadzono dopasowanie metodą najmniejszych kwadratów krzywej-s.

\begin{figure}
        \centering
        \begin{multicols}{2}
                \includegraphics[width=0.5\textwidth]{scurve/test_1A.jpg} \par
                \includegraphics[width=0.5\textwidth]{scurve/test_2A.jpg} \par
        \end{multicols}
        \caption{Krzywe-S dla testów z wykorzystaniem sygnałów testowych.}\label{s curve test}
\end{figure}

Funkcja wykorzystana do fitowania krzywej przybrała następującą formę:
\begin{equation}
        \label{test eq}
        S(x) = \frac{f_g}{2} * erf^{-1}(\frac{x-\overline{x}}{\sigma*\sqrt{2}})
\end{equation}
Gdzie:
\begin{description}
        \item $f_g$ - częstotliwość generatora,
        \item $erf^{-1}$ - odwrotna funkcja błędu
        \item $x$ - napięcie dyskryminatora,
        \item $\overline{x}$ - poziom dyskryminacji odpowiadający średniej amplitudzie impulsów (peak position) ,
        \item  $\sigma$ - wartość niepewności odpowiadająca wartości szumowej układu. 
\end{description}

Dopasowywane współczynniki to $f_g$, $\overline{x}$ oraz $\sigma$. Na podstawie wartości $\sigma$ oraz $\overline{x}$ obliczona została wartość ENC. Użyta równość znajduje się poniżej:
\begin{equation}
        ENC [e^-] = \frac{\sigma}{\overline{x}} * \frac{E_{fe}}{E_{(-,+)}}
\end{equation}
Gdzie:
\begin{description}
        \item $ENC$  - równoważny ładunek szumowy,
        \item $E_{(-,+)}$ - energia potrzebna na wytworzenie jednej pary dziura/elektron w krzemie (3.6eV),
        \item $\overline{x}$ - poziom dyskryminacji odpowiadający średniej amplitudzie impulsów (peak position) dla pomiaru ze źródłem izotopowym dla źródła ${}^{55}Fe$ (dział \ref{section RXHDR fe})
        \item $E_{fe}$ -  energia deponowana w krzemie przez źródło ${}^{55}Fe$  (5.9keV)
\end{description}

Wartości tych współczynników znajdują się w tabeli \ref{tabela wsp test} (Aneks) a wykresy tych wartości po kanałach znajdują się na grafice \ref{test fit wsp wyk} 

\begin{figure}[b]
        \begin{multicols}{2}
                \includegraphics[width=0.5\textwidth]{esptest/CzTest.png} \par
                \includegraphics[width=0.5\textwidth]{esptest/ENCTest.png} \par       
        \end{multicols} \hfill
        \begin{multicols}{2}
                \includegraphics[width=0.5\textwidth]{esptest/OdchylTest.png} \par
                \includegraphics[width=0.5\textwidth]{esptest/PoziomDTest.png} \par
        \end{multicols}
        \caption{Wartości współczynników dopasowanych do krzywych-S \ref{s curve test}}
        \label{test fit wsp wyk}
\end{figure}





\subsubsection{Testowanie systemu liczników przy użyciu układu RXHDR\_V1 oraz źródła promieniotwórczego}
\label{section RXHDR fe}

Przeprowadzono badanie analogicznie do tych w dziale \ref{section RXHDR test}, jednak zamiast impulsów wprowadzanych na wejście testowe układu RXHDR\_V1 sygnał wytwarzany był w sensorze promieniowania przy oświetlaniu źródłem promieniowania ${}^{55}Fe$ ustawionym bezpośrednio nad oknem detektora. 

Wykresy zależności napięcia dyskryminatora względem liczby zliczeń umieszczone zostały na wykresie \ref{s curve fe}. 

\begin{figure}[]
        \centering
        \begin{multicols}{2}
                \includegraphics[width=0.5\textwidth]{scurve/fe_1A.jpg} \par
                \includegraphics[width=0.5\textwidth]{scurve/fe_2A.jpg} \par
                
        \end{multicols}
        \caption{Krzywe-S dla testów z wykorzystaniem źródła promieniotwórczego ${}^{55}Fe$}
        \label{s curve fe}
\end{figure}

Do wykresów na rysunku \ref{s curve fe} dopasowana została krzywa-S z uwzględnieniem współczynnika podziału ładunku. Wzór przybiera następującą postać:

\begin{equation}
        \label{test eq}
        S(x) = (1-RCS * (\frac{x}{\overline{x}}-0.5)) * \frac{f_g}{2} * erf^{-1}(\frac{x-\overline{x}}{\sigma*\sqrt{2}})
\end{equation}
Gdzie:
\begin{description}
        \item $f_g$ - średnia częstość rejestrowanych przypadków detekcji,
        \item $erf^{-1}$ - odwrotna funkcja błędu
        \item $x$ - napięcie dyskryminatora,
        \item $\overline{x}$ - poziom dyskryminacji odpowiadający średniej amplitudzie impulsów (peak position) ,
        \item  $\sigma$ - wartość niepewności odpowiadająca wartości szumowej układu,
        \item $RCS$ -  współczynnik podziału ładunku.
\end{description}

Na podstawie wartości $\sigma$ oraz $\overline{x}$ wyliczona została wartość ENC:

\begin{equation}
        ENC [e^-] = \frac{\sigma}{\overline{x}} * \frac{E_{fe}}{E_{(-,+)}}
\end{equation}

Wartości otrzymanych współczynników wraz z niepewnościami umieszczone zostały w tabeli~\ref{tabela wsp fe} (Aneks). Wykresy wartości tych współczynników po kanałach znajdują się na rysunku \ref{wyk wsp fe}.

Średnia wartość $\overline{x}$ dla wszystkich kanałów została policzona i wynosi $191.02 \pm 0.17 [mV]$




\begin{figure}
        \begin{multicols}{2}
                \includegraphics[width=0.5\textwidth]{esptest/CzFe.png} \par
                \includegraphics[width=0.5\textwidth]{esptest/ENCFe.png} \par       
        \end{multicols} \hfill
        \begin{multicols}{2}
                \includegraphics[width=0.5\textwidth]{esptest/OdchylFe.png} \par
                \includegraphics[width=0.5\textwidth]{esptest/PoziomDFe.png} \par
        \end{multicols}
        \caption{Wartości współczynników dopasowanych do krzywych-S \ref{s curve fe}}
        \label{wyk wsp fe}
\end{figure}
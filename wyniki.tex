\section{Wyniki badań}

Poniżej umieszczone są wyniki badań z zastosowaniem użytych rozwiązań. 

\subsection{Rozwiązanie softwarowe na platformie Arduino}

Przeprowadzono testy polegające na bezpośrednim zliczaniu sygnałów generowanych przez generator zewnętrzy. 
Napięciem, kształtem oraz długością odpowiadały tym produkowanym przez układ RXHDR\_V1 \cite{master}.

Przeprowadzono po pięć badań na każdą częstotliwość i wyliczono błąd względem spodziewanego wyniku. 
Czas akwizycji ustalono na jedną sekundę sprawiając że liczba spodziewanych zliczeń jest równa częstotliwości generowanych sygnałów.

Wyniki znajdujące się w tabeli \ref{rts table} zwizualizowane są na wykresie \ref{rts wyniki}.


\begin{figure}[]
        \centering
        \includegraphics[width=\textwidth]{rts.jpg}
        \caption{Wyniki testów z wykorzystaniem przerwań systemowych}
        \label{rts wyniki}
\end{figure}

\begin{table}
        \centering
        \caption{Wyniki rozwiązania z zastosowaniem przerwań systemowych}
        \label{rts table}
        \begin{tabular}{|c|c|c|c|c|}  
                \hline 
                Częstotliwość [kHz] & Estymowana ilość & Otrzymana ilość & Różnica  & Błąd względny [\%]\\ 
                &  zliczeń &  zliczeń & w zliczeniach & \\ \hline
                1 & 1000 & 1001 & 1 & 0.1000\\ \hline 
                1.414 & 1414 & 1415 & 1 & 0.0707\\ \hline 
                2 & 2000 & 2001 & 1 & 0.0500\\ \hline 
                2.82 & 2820 & 2820 & 0 & 0.0000\\ \hline 
                4 & 4000 & 3999 & 1 & 0.0250\\ \hline 
                5.65 & 5650 & 5650 & 0 & 0.0000\\ \hline 
                8 & 8000 & 7996 & 4 & 0.0500\\ \hline 
                11.3 & 11300 & 11298 & 2 & 0.0177\\ \hline 
                16 & 16000 & 15989 & 11 & 0.0688\\ \hline 
                22.62 & 22620 & 22605 & 15 & 0.0663\\ \hline 
                32 & 32000 & 31986 & 14 & 0.0438\\ \hline 
                45.25 & 45250 & 45230 & 20 & 0.0442\\ \hline 
                64 & 64000 & 63948 & 52 & 0.0813\\ \hline 
                90.5 & 90500 & 90499 & 1 & 0.0011\\ \hline 
                128 & 128000 & 127943 & 57 & 0.0445\\ \hline 
                181 & 181000 & 180874 & 126 & 0.0696\\ \hline 
                256 & 256000 & 255758 & 242 & 0.0945\\ \hline 
                362 & 362000 & 361962 & 38 & 0.0105\\ \hline 
                512 & 512000 & 511920 & 80 & 0.0156\\ \hline 
                700 & 700000 & 699937 & 63 & 0.0090\\ \hline 
                710 & 710000 & 709686 & 314 & 0.0442\\ \hline 
                715 & 715000 & 714941 & 59 & 0.0083\\ \hline
        \end{tabular}
\end{table}

Wyniki pokazują dokładność mieszczącą się w  1\textperthousand, 
jednak błąd w otrzymanym wyniku zależy od częstotliwości generowanych zliczeń.
Może to być spowodowane interferencją przerwania powodującego zliczenie z innymi przerwaniami wymaganymi do działania mikrokontrolera.

Po osiągnięciu częstotliwości granicznej > 0.715 [MHz] program mikrokontrolera przestaje wysyłać dane na komputer zewnętrzny.
Jest to spowodowane tym że zaraz po wyjściu z obsługi przerwania systemowego program natychmiast zaczyna obsługę następnego przerwania. 
Liczba graniczna pozwala przybliżyć czas konieczny na wykonanie jednego przerwania na podstawie przekształcenia wzoru \ref{Cykli w sec}. 
$$ t_p = \frac{1}{f_p} = \sim 1.3986 [\mu s] $$
$$ N_c = \frac{t_p}{t_c} =  \frac{1.3986 \mu s}{11.9 ns} =\sim 118$$
gdzie: \\
        \indent $t_p$ -  czas potrzebny na obsługą przerwania\\
        \indent $t_c$ -  czas jednego cyklu procesora (dział \ref{dzial arduino} ) \\
        \indent $f_p$ -  częstotliwość graniczna przerwań \\
        \indent $N_c$ -  ilość cykli koniecznych na pojedyncze przerwanie \\

Liczba cykli na przerwanie jest mniejsza niż ta szacowana w dziale \ref{dzial arduino} (355 + 128)\cite{ard_opt_git}, wynika to z faktu że poprzednia estymacja była wykonana dla najgorszego przypadku dla nieoptymalizowanego kodu.
Mimo lepszych osiągów niż te szacowane wynik ten nadal odbiega od optymalnego czasu wywołania przerwania (12 + 10) \cite{interupt latency} i wciąż jest znacznie poniżej wymagań projektu. 

Dodatkowe testy potwierdziły że wraz z zwiększeniem ilości badanych kanałów częstotliwość graniczna zmniejsza się jak $\frac{1}{n}$ gdy $n$ to ilość badanych kanałów. 

\subsection{Rozwiązanie z układem zewnętrznych liczników buforujących}

Dane do poniższych wyników zostały uzyskane przez analizę plików archiwalnych otrzymywanych przy działaniu programu do kontroli układu.
Pliki te są przechowywane w formacie JSON.

Analiza sygnałów na została dokonana na oscyloskopie keysight 3024A. Generator sygnałów używany w układzie badawczym to generator tektonix AFG3102.

Sygnał używany do symulowania impulsów układu RXHDR\_v1 miał własności: 1.8V, kształt prostokątny, impuls o szerokości 50 us.


\subsubsection{Kalibracja układu}
\label{section kaliblracja}

Przed przystąpieniem do końcowych testów układu konieczne jest przeprowadzenie kalibracji ze względu na czas martwy układu. 
Oczekiwanym efektem jest regulacja czasu akwizycji w taki sposób że liczba zliczeń będzie odpowiadać ilości impulsów która powinna być uzyskana w czasie akwizycji przez układ bez czasu martwego. 

Zbadano okres jednego cyklu pomiaru (8 kanałów) i uzyskano czas cyklu równy $4.52 \mu s$. Taki cykl jest widoczny na rysunku \ref{Oscyloskop}.
Na podstawie czasu liczby można wyliczyć ilość cykli akwizycji przypadających na pojedyńcza mikrosekundę.

\begin{equation}
        \label{per cykl}
        CIRCLES\_FOR\_1MS = \frac{1*10^{-3}}{4.52 * 10^{-6}} = 221.238938053
\end{equation}

Korzystając z tej wartości i wartości współczynnika korekcji czasu martwego równego 1 przeprowadzono badania zależności częstotliwości podawanej z generatora do częstotliwości zliczeń dla różnych czasów akwizycji.  

Układ badawczy składał się z komputera, Arduino due, badanego układu elektronicznego, taśmy transmisyjnej, oscyloskopu, generatora impulsów, zasilacza. 

Przeprowadzono trzy badania dla 10 częstotliwości dla 2 czasów akwizycji. Na podstawie tych danych przeprowadzono fitowanie krzywej $f(x) = a*x$ wyniki fitowania umieszczone zostały w tabeli poniżej.

\begin{table}
        \caption{Wyniki fitowania konieczne dla ustalenia korekcji czasu martwego}
        \label{dead time fit}
        \centering
        \begin{tabular}{|c|c|c|c|c|}
                \hline
                Nazwa kanału & czas akwizycji & $a$ & $\Delta a$ & konwersja na 1s \\ \hline
                1A6 & 1000 ms & 0.879395 & $1.893 * 10^{-5}$ & 0.879395 \\ \hline
                2A5 & 500 ms & 0.440366 & $8.156 * 10^{-6}$ & 0.880732 \\ \hline
                2A7 & 500 ms & 0.440352 & $7.939 * 10^{-6}$ & 0.880704 \\ \hline
        \end{tabular}
\end{table}

Na podstawie powyższych danych ustalono wartość współczynnika DEAD\_TIME\_CORECTION równe:
\begin{equation}
        \label{dead time eq}
        DEAD\_TIME\_CORECTION = \frac{1}{\sum^n_i \frac{a_i}{n}} = 1.13636
\end{equation} 

\begin{figure}
        \centering
        \begin{multicols}{2}
                \includegraphics[width=0.5\textwidth]{dead1A6.jpg} \par
                \includegraphics[width=0.5\textwidth]{dead2A7.jpg} \par
        \end{multicols}
        \caption{Wybrane wykresy użyte w celu kalibracji układu. Czas akwizycji 1000 ms lewy wykres oraz 500 ms prawy wykres}
        \label{wykresy fit calib}
\end{figure}

Zbadano dodatkowo maksymalny częstotliwość dla której następuje przepełnienie liczników i jest to wartość $3.72 [MHz]$. Po przekroczeniu tej wartości częstotliwości generatora wartość zliczeń drastycznie spada. 

\paragraph{}{Błąd przybliżenia do liczby naturalniej}
Jak widać w fragmencie kodu \ref{code aqw} liczba cykli przeprowadzonych w trakcie pojedyńczej akwizycji musi być liczbą całkowitą. Liczba cykli jest określona wzorem:
\begin{equation}
        N_c [-] = int(A_t*D_t*f_c)
\end{equation}
Gdzie:
\begin{description}
        \item $N_c$ - liczba cykli w procesie akwizycji,
        \item $A_t$ - czas  akwizycji w ms
        \item $D_t$ - współczynnik korekcji czasu martwego 
        \item $f_c$ - ilość cykli w ms
\end{description}

Błąd wprowadzony w wyniku przybliżenia mieści się w przedziale wartości [0,1) pozwala to na wyliczenie maksymalnego względnego błędu uzyskiwanego w wyniku przybliżania:
\begin{equation}
        \Delta N_{c_{max}} [\%] = \frac{1}{A_t*D_t*f_c}  * 100\%
\end{equation} 
Gdzie:
\begin{description}
        \item $\Delta N_{c_{max}}$ - maksymalny względny błąd,
        \item $A_t$ - czas  akwizycji w ms
        \item $D_t$ - współczynnik korekcji czasu martwego 
        \item $f_c$ - ilość cykli w ms
\end{description}

Podstawiając współczynniki uzyskane w dziale \ref{section kaliblracja} możemy uzyskać wartości maksymalnego błędu dla najczęściej używanych czasów akwizycji. Wartości te znajdują się w tabeli \ref{tab przyblizenie niep}. Na podstawie tych wartości można stwierdzić że błąd popełniony podczas przybliżania zmienia się w zależności od czasu akwizycji jest on jednak pomijalne mały.

\begin{table}
        \centering
        \caption{Wartości błędu wynikające z przybliżenia liczbą całkowitą}
        \label{tab przyblizenie niep}
        \begin{tabular}{|c|c||c|c|}
                \hline
                Czas akwizycji [ms] &   $\Delta N_{c_{max}} [\%]$&Czas akwizycji [ms] &   $\Delta N_{c_{max}} [\%]$ \\ \hline
                50 & 0.007955 & 100 & 0.003977 \\ \hline
                1000 & 0.0003977 & 5000 & 0.00007955 \\ \hline
        \end{tabular}
\end{table}

\paragraph{Ograniczenie w czasu akwizycji}

Konieczne jest rozważenie problemu przetrzymywania danych zliczeń dla kanału w pojedyńczej zmiennej. Teoretyczny czas akwizycji jest nieskończony jednak w pewnej chwili wartość zmiennej przechowującej dane o zliczeniu zostanie przepełniona i wartość się zmniejszy.

W celu przechowywania zmiennej zawierającej informacje o ilości zliczeń użyty została zmienna typu \textit{unsigned int} dla mikrokontrolera AT91SAM3X8E wartość ta mieści się w 32 bitach. Fakt że zmienna ta nie ma znaku sprawia że wartości graniczne tej liczby to [ 0, $2^{33}-1$ ] oznacza to że dla zbadanej maksymalnej częstotliwości granicznej układu 3.72 [MHz] możemy wyliczyć czas akwizycji dla której nastąpi przepełnienie dla maksymalnej częstotliwości zliczeń. 
\begin{equation}
        A_{t_{max}} = \frac{N_{max}}{f_{max}} = \frac{2^{33}-1}{3.72 * 10^{6}} = \frac{8589934591}{3.72 * 10^{6}} = 2309.12 [s] = ~6.5 [h]
\end{equation}
Gdzie:
\begin{description}
        \item $A_{t_{max}}$ - maksymalny czas akwizycji ze względu na efekt przepełnienia liczby 32 bitowej
        \item $N_{max}$ - maksymalna wartość liczby 32 bitowej
        \item $f_{max}$ - maksymalna częstotliwość układu liczników
\end{description}

Wartość ta jest absurdalnie duża i nieprzydatna ponieważ przez cały czas akwizycji mikorkontroler jest nieresponsywny dlatego też komputer kontrolujący już po 5 s od wysłania zapytania zakłada że na kontrolerze wystąpił błąd i rozpoczyna proces wysyłania zapytań w celu zbadania problemu. Sprawia to że ze względu tą decyzję maksymalna wartość czasu akwizycji równa się 10 s. Jeżeli konieczne jest zwiększenie tej liczby należy zwiększyć wartość zmiennej WAIT\_FOR. Problem ten nie dotyczy funkcjonalności \textit{Single Shoot}.   

\subsubsection{Testowanie układu liczników za pomocą generatora zewnętrzego.}

Przy użyciu współczynników otrzymanych w dziale \ref{section kaliblracja} oraz układu badawczego odpowiadającemu temu z działu \ref{section kaliblracja} przeprowadzono test układu z wykorzystaniem dzielnika pozwalającego na dostarczenie sygnałów impulsów, odpowiadającym tym generowanym przez układ RXHDR\_V1, do wszystkich kanałów jednocześnie.
Badanie przeprowadzono dla 10 częstotliwości od 100Hz do 3.2Mhz. Czas akwizycji badania był równy 1s dzięki czemu liczba zliczeń odpowiada liczbie zliczeń. Uśrednione wartości wyników dla wybranych kanałów znajdują się w tabeli \ref{tabela wyniki surowe}.

\begin{figure}
        \begin{multicols}{2}
            \includegraphics[width=0.5\textwidth]{HM1KHz.jpg} \par    
            \includegraphics[width=0.5\textwidth]{HM1MHz.jpg} \par    
        \end{multicols} \hfill
        \begin{multicols}{2}
            \includegraphics[width=0.5\textwidth]{H1A532KHz.jpg} \par
            \includegraphics[width=0.5\textwidth]{H1A81KHz.jpg} \par    
        \end{multicols}
        \caption{Histogramy rozrzutu zliczeń dla wybranych kanałów i częstotliwości}
        \label{hist licz}
\end{figure}



Każdy z punktów pomiarowych składał się z serii osobnych akwizycji. Jako wartość punktu pomiarowego ustalono średnią z pomiaru, a jako niepewność pomiaru (zaznaczona na wykresach i uwzględniona w dopasowaniu krzywej) jako odchylenie standardowe średniej. Wartości częstotliwości pojedynczych pomiarów dla wszystkich kanałów dla wybranych częstotliwości znajdują się na histogramach zawartych na rysunkach \ref{hist licz}. 

Wyniki odchylenia standardowego oraz różnicy między podawaną częstotliwością a liczbą zliczeń umieszczone są na rysunku \ref{3d licznik}. Wartości te podawane są jako wartości bezwzględne oraz wartości związane z częstotliwością generowanych sygnałów.

\begin{figure}
        \centering
        \begin{multicols}{2}
                \includegraphics[width=0.5\textwidth]{3d/difLiczik.jpg} \par                
                \includegraphics[width=0.5\textwidth]{3d/niepLiczik.jpg} \par                
        \end{multicols} \hfill
        \begin{multicols}{2}
                \includegraphics[width=0.5\textwidth]{3d/difWzgLiczik.jpg} \par                
                \includegraphics[width=0.5\textwidth]{3d/niepWzgLiczik.jpg} \par                
        \end{multicols}
        \caption{Wartości odchylenia standardowego oraz różnica między zliczeniami podawane w wartościach bezwzględnych oraz jako wartości względne. }
        \label{3d licznik}
\end{figure}

Wybrane wykresy pomiaru zależności częstotliwości od liczby zliczeń umieszczone są na rysunku \ref{multi wyk} a wyniki fitowania znajdują się w tabeli \ref{multi fit}. Wyniki pomiarów fitowane były krzywą $f(x) = a*x$. 
Średnia po wszystkich współczynnikach to $0.999997 \pm 2.56*10^{-5}$
odchylenie standardowe wyników jest równe  $3.94 * 10^{-5}$.

\begin{figure}
        \centering
        \begin{multicols}{2}
                \includegraphics[width=0.5\textwidth]{multi1A3.jpg} \par
                \includegraphics[width=0.5\textwidth]{multi1A6.jpg} \par                
        \end{multicols} \hfill
        \begin{multicols}{2}
                \includegraphics[width=0.5\textwidth]{multi2A2.jpg} \par
                \includegraphics[width=0.5\textwidth]{multi2A7.jpg} \par                
        \end{multicols}
        \caption{Wybrane wykresy testów układu licznika wraz z dopasowanymi krzywymi liniowymi}
        \label{multi wyk}
\end{figure}

\begin{table}
        \centering
        \caption{Wyniki fitowania krzywej $f(x) = a*x$ dla pomiarów generowanych z generatora}
        \label{multi fit}
        \begin{tabular}{|c|c|c||c|c|c|}
                \hline
                kanał & $a$ & $\Delta a$  &kanał & $a$ & $\Delta a$ \\ \hline
                1A1 & 1.00004 & 1.101e-05 & 2A1 &1.00006&1.129e-05 \\ \hline
                1A2 & 0.999961 &2.649e-05& 2A2 &0.999956&4.27e-05 \\ \hline
                1A3 & 0.999988&2.155e-05&2A3 & 1.00003&2.808e-05\\ \hline
                1A4 & 0.999951&2.792e-05&2A4&0.999951&4.309e-05\\ \hline
                1A5&1.00004&1.323e-05&2A5&1.00006&1.223e-05\\ \hline
                1A6&1&2.467e-05&2A6&1.00002&2.792e-05 \\ \hline
                1A7&0.999979&2.002e-05&2A7&1.00001&2.798e-05 \\ \hline
                1A8 &0.999949&2.747e-05&2A8&0.999957&4.373e-05 \\ \hline
        \end{tabular}
\end{table}

\begin{table}
        \centering
        \caption{Wyniki badania zależności częstotliwości od liczby zliczeń dla czasu akwizycji 1s dla wybranych kanałów}
        \label{tabela wyniki surowe}
        \begin{multicols}{3}
                \begin{adjustbox}{width=1\linewidth,left}
                \begin{tabular}{|c|c|c|}  
                        \hline 
                        1A1 & & \\ \hline
                        f [Hz] & zliczenia & $\Delta$ zliczeń \\ \hline
                       100 & 99.68 & 2.76\\ \hline 
                       320 & 319.72 & 4.27\\ \hline 
                       1000 & 1001.53 & 4.20\\ \hline 
                       3200 & 3201.92 & 4.02\\ \hline 
                       10000 & 10006.04 & 7.35\\ \hline 
                       32000 & 32022.43 & 14.02\\ \hline 
                       100000 & 100053.84 & 19.37\\ \hline 
                       320000 & 320206.69 & 79.24\\ \hline 
                       1000000 & 1000166.75 & 51.99\\ \hline 
                       3200000 & 3200122.67 & 22.92\\ \hline
               \end{tabular}
               \end{adjustbox}
               \begin{adjustbox}{width=1\linewidth,left}

                \begin{tabular}{|c|c|c|} 
                        \hline 
                        1A3 & & \\ \hline
                        f [Hz] & zliczenia & $\Delta$ zliczeń \\ \hline
                       100 & 100.27 & 2.59\\ \hline 
                       320 & 319.39 & 4.09\\ \hline 
                       1000 & 999.75 & 0.98\\ \hline 
                       3200 & 3203.41 & 6.79\\ \hline 
                       10000 & 10005.45 & 7.13\\ \hline 
                       32000 & 32019.90 & 17.58\\ \hline 
                       100000 & 100050.45 & 18.87\\ \hline 
                       320000 & 320202.57 & 79.36\\ \hline 
                       1000000 & 1000187.97 & 56.24\\ \hline 
                       3200000 & 3199920.11 & 38.53\\ \hline
               \end{tabular}
        \end{adjustbox}

               \begin{adjustbox}{width=1\linewidth,left}

               \begin{tabular}{|c|c|c|}  
                \hline 
                2A7 & & \\ \hline
                f [Hz] & zliczenia & $\Delta$ zliczeń \\ \hline
                100 & 100.43 & 2.46\\ \hline 
                320 & 319.94 & 4.01\\ \hline 
                1000 & 1000.56 & 4.56\\ \hline 
                3200 & 3203.77 & 5.77\\ \hline 
                10000 & 10004.45 & 6.20\\ \hline 
                32000 & 32020.41 & 17.16\\ \hline 
                100000 & 100045.22 & 20.15\\ \hline 
                320000 & 320061.73 & 19.52\\ \hline 
                1000000 & 1000083.13 & 25.85\\ \hline 
                3200000 & 3199841.89 & 56.18\\ \hline
        \end{tabular} \par
\end{adjustbox}

        \end{multicols}

\end{table}




\subsubsection{Testowanie układu liczników przy użyciu kanału testowego układu RXHDR\_V1}
\label{section RXHDR test}

Układ badawczy składał się z komputera, Arduino due, badanego układu elektronicznego, taśmy transmisyjnej, oscyloskopu, generatora impulsów, zasilacza, układu RXHDR\_V1, zasilacza wysokiego napięcia. 

Układ RXHDR\_v1 połączony z zestawem liczników buforujących został zasilony napięciem 3.6V oraz 1.8V a detektor został spolaryzowany przez zasilacz wysokiego napięci napięciem 300V. Następnie zestaw ten został połączony z Arduino Due i komputerem kontrolującym. 

Generator połączony z wejściem testowym układu RXHDR\_V1 generował sygnał prostokątny o napięciu 0.3V i częstotliwości 100KHz. 

Pomiar polegał na akwizycji danych (1000ms) dla różnych napięć dyskryminatora w celu utworzenia krzywej S. Wykresy tych pomiarów umieszczone zostały na wykresach \ref{s curve test}. Na podstawie tych wykresów przeprowadzono dopasowanie metodą najmniejszych kwadratów krzywej s.

\begin{figure}
        \centering
        \begin{multicols}{2}
                \includegraphics[width=0.5\textwidth]{scurve/test_1A.jpg} \par
                \includegraphics[width=0.5\textwidth]{scurve/test_2A.jpg} \par
        \end{multicols}
        \caption{Krzywe s dla badań z wykorzystaniem sygnałów testowych.}\label{s curve test}
\end{figure}

Funkcja wykorzystana do fitowania krzywej przybrała następującą formę:
\begin{equation}
        \label{test eq}
        S(x) = \frac{f_g}{2} * erf^{-1}(\frac{x-\overline{x}}{\sigma*\sqrt{2}})
\end{equation}
Gdzie:
\begin{description}
        \item $f_g$ - częstotliwość generatora,
        \item $erf^{-1}$ - odwrotna funkcja błędu
        \item $x$ - napięcie dyskryminatora,
        \item $\overline{x}$ - poziom dyskryminacji odpowiadający średniej amplitudzie impulsów (peak position) ,
        \item  $\sigma$ - wartość niepewności odpowiadająca wartości szumowej układu. 
\end{description}

Dopasowywane współczynniki to $f_g$, $\overline{x}$ oraz $\sigma$. Na podstawie wartości $\sigma$ oraz $\overline{x}$ obliczona została wartość ENC. Użyta równość znajduje się poniżej:
\begin{equation}
        ENC [e^-] = \frac{\sigma}{\overline{x}} * \frac{E_{fe}}{E_{(-,+)}}
\end{equation}
Gdzie:
\begin{description}
        \item $ENC$  - równoważny ładunek szumowy,
        \item $E_{(-,+)}$ - energia wymagana na wytworzenie pary dziura/elektron w krzemie,
        \item $\overline{x}$ - poziom dyskryminacji odpowiadający średniej amplitudzie impulsów (peak position) dla źródła ${}^{55}Fe$ (dział \ref{section RXHDR fe})
        \item $E_{fe}$ -  energia deponowana w krzemie przez źródło ${}^{55}Fe$
\end{description}

Wartości tych współczynników znajdują się w tabeli \ref{tabela wsp test} a wykresy tych wartości po kanałach znajdują się na grafice \ref{test fit wsp wyk} 

\begin{figure}
        \begin{multicols}{2}
                \includegraphics[width=0.5\textwidth]{scurve/Generowana_czestotliwosc_fit.jpg} \par
                \includegraphics[width=0.5\textwidth]{scurve/srednia_fit.jpg} \par       
        \end{multicols} \hfill
        \begin{multicols}{2}
                \includegraphics[width=0.5\textwidth]{scurve/odchylenie_fit.jpg} \par
                \includegraphics[width=0.5\textwidth]{scurve/ENC_fit_test.jpg} \par
        \end{multicols}
        \caption{Wartości współczynników dopasowanych do krzywych s \ref{s curve test}}
        \label{test fit wsp wyk}
\end{figure}


\begin{table}
        \caption{Tabela współczynników testu z użyciem sygnałów testowych, dopasowanych do krzywych s na rysunku \ref{s curve test}}
        \label{tabela wsp test}
        \begin{adjustbox}{width=1.1\linewidth,center}
        \begin{tabular}{|c|c|c|c|c|c|c|c|c|}
                \hline
                kanał & $f_g [Hz]$&$\Delta f_g [Hz]$&$\overline{x} [mV]$&$\Delta \overline{x} [mV]$&  $\sigma$ [mV]&  $\Delta \sigma [mV]$ & ENC $[e^-_{rms}]$& $\Delta$ ENC $[e^-_{rms}]$\\ \hline

                        1A1&100058.81&8.80&320.73&0.06&9.88&0.05&83.84&0.45 \\ \hline 
                        1A2&100061.52&8.83&324.99&0.04&10.60&0.05&89.93&0.42 \\ \hline 
                        1A3&100066.02&7.97&322.72&0.04&10.19&0.05&86.48&0.41 \\ \hline 
                        1A4&100057.83&8.15&327.79&0.05&10.95&0.06&92.89&0.48 \\ \hline 
                        1A5&100074.82&8.80&325.27&0.03&10.92&0.05&92.63&0.39 \\ \hline 
                        1A6&100068.24&9.27&322.14&0.04&10.39&0.05&88.17&0.42 \\ \hline 
                        1A7&100063.50&8.47&328.92&0.05&11.45&0.06&97.14&0.48 \\ \hline 
                        1A8&100028.93&8.85&317.58&0.04&9.30&0.05&78.88&0.40 \\ \hline 
                        2A1&100059.06&8.65&331.45&0.05&11.71&0.06&99.33&0.50 \\ \hline 
                        2A2&100054.34&8.28&322.21&0.05&10.00&0.05&84.84&0.46 \\ \hline 
                        2A3&100061.20&7.59&327.16&0.04&11.53&0.06&97.82&0.49 \\ \hline 
                        2A4&100054.62&8.78&324.77&0.06&10.82&0.07&91.80&0.59 \\ \hline 
                        2A5&100058.74&8.20&324.23&0.04&10.35&0.05&87.79&0.45 \\ \hline 
                        2A6&100055.83&8.27&332.20&0.04&11.69&0.05&99.20&0.42 \\ \hline 
                        2A7&100058.83&8.09&328.71&0.05&11.31&0.06&95.93&0.52 \\ \hline 
                        2A8&100041.80&7.66&325.00&0.04&10.99&0.05&93.25&0.44 \\ \hline                    
        \end{tabular}
        \end{adjustbox} 
\end{table}


\subsubsection{Testowanie układu liczników przy użyciu układu RXHDR\_V1 oraz żródła promieniotwórczego}
\label{section RXHDR fe}

Przeprowadzono badanie analogicznie do tych w dziale \ref{section RXHDR test}, jednak zamiast impulsów wprowadzanych na wejście testowe układu RXHDR\_V1 sygnał generowany był przez źródło promieniotwórcze ${}^{55}fe$ ustawione bezpośrednio nad oknem detektora. 

Wykresy zależności napięcia dyskryminatora względem liczby zliczeń umieszczone zostały na wykresie \ref{s curve fe}. 

\begin{figure}[]
        \centering
        \begin{multicols}{2}
                \includegraphics[width=0.5\textwidth]{scurve/fe_1A.jpg} \par
                \includegraphics[width=0.5\textwidth]{scurve/fe_2A.jpg} \par
                
        \end{multicols}
        \caption{Krzywe s dla badań z wykorzystaniem źródła promieniotwórczego}
        \label{s curve fe}
\end{figure}

Do wykresów na rysunku \ref{s curve fe} dopasowana została krzywa s z uwzględnieniem współczynnika podziału ładunku. Wzór przybiera następującą postać:

\begin{equation}
        \label{test eq}
        S(x) = (1-RCS * (\frac{x}{\overline{x}}-0.5)) * \frac{f_g}{2} * erf^{-1}(\frac{x-\overline{x}}{\sigma*\sqrt{2}})
\end{equation}
Gdzie:
\begin{description}
        \item $f_g$ - częstotliwość generatora,
        \item $erf^{-1}$ - odwrotna funkcja błędu
        \item $x$ - napięcie dyskryminatora,
        \item $\overline{x}$ - poziom dyskryminacji odpowiadający średniej amplitudzie impulsów (peak position) ,
        \item  $\sigma$ - wartość niepewności odpowiadająca wartości szumowej układu,
        \item $RCS$ -  współczynnik podziału ładunku.
\end{description}

Na podstawie wartości $\sigma$ oraz $\overline{x}$ wyliczona została wartość ENC:

\begin{equation}
        ENC [e^-] = \frac{\sigma}{\overline{x}} * \frac{E_{fe}}{E_{(-,+)}}
\end{equation}
Gdzie:
\begin{description}
        \item $ENC$  - równoważny ładunek szumowy,
        \item $E_{(-,+)}$ - energia wymagana na wytworzenie pary dziura/elektron w krzemie,
        \item $\overline{x}$ - poziom dyskryminacji odpowiadający średniej amplitudzie impulsów (peak position) dla fitowanego kanału,
        \item $E_{fe}$ -  energia deponowana w krzemie przez źródło ${}^{55}Fe$
\end{description}

Wartości otrzymanych współczynników wraz z niepewnościami umieszczone zostały w tabeli \ref{tabela wsp fe}. Wykresy wartości tych współczynników po kanałach znajdują się na rysunku \ref{wyk wsp fe}.

Średnia wartość $\overline{x}$ dla wszystkich kanałów została policzona i wynosi $191.02 \pm 0.17 [mV]$


\begin{table}
        \centering
        \caption{Tabela współczynników testu z użyciem źródła promieniotwórczego, dopasowanych do krzywych s na rysunku \ref{s curve fe}}
        \label{tabela wsp fe}
        \begin{adjustbox}{width=1.1\linewidth,center}
                \begin{tabular}{|c|c|c|c|c|c|c|c|c|c|c|}
                \hline
                kanał & $f_g [Hz]$&$\Delta f_g [Hz]$&$\overline{x} [mV]$&$\Delta \overline{x} [mV]$&  $\sigma$ [mV]&  $\Delta \sigma [mV]$ &RCS&$\Delta$ RCS& ENC $[e^-_{rms}]$& $\Delta$ ENC $[e^-_{rms}]$\\ \hline
                1A1&20173.65&52.14&185.81&0.17&15.14&0.14&0.44&0.01&132.10&1.27  \\ \hline 
                1A2&20344.09&51.38&191.21&0.16&14.82&0.14&0.47&0.01&125.65&1.19  \\ \hline 
                1A3&20403.52&45.63&188.55&0.15&15.27&0.13&0.45&0.01&131.27&1.11  \\ \hline 
                1A4&20379.81&43.01&193.48&0.16&15.37&0.12&0.46&0.01&128.80&1.01  \\ \hline 
                1A5&20315.39&46.61&188.60&0.19&15.61&0.14&0.47&0.01&134.15&1.19  \\ \hline 
                1A6&20385.95&47.43&188.80&0.18&15.19&0.13&0.47&0.01&130.37&1.12  \\ \hline 
                1A7&20458.17&44.85&195.31&0.17&15.34&0.14&0.49&0.01&127.34&1.17  \\ \hline 
                1A8&20277.88&46.78&184.63&0.20&15.08&0.15&0.47&0.01&132.43&1.30  \\ \hline 
                2A1&20336.26&45.10&195.12&0.17&15.26&0.14&0.50&0.01&126.75&1.17  \\ \hline 
                2A2&20433.52&48.91&189.32&0.19&15.17&0.15&0.47&0.01&129.86&1.26  \\ \hline 
                2A3&20285.03&44.09&190.34&0.19&16.07&0.15&0.48&0.01&136.87&1.26  \\ \hline 
                2A4&20433.73&43.03&191.42&0.18&15.24&0.15&0.47&0.01&129.03&1.32  \\ \hline 
                2A5&20607.82&42.85&190.80&0.17&14.96&0.15&0.48&0.01&127.09&1.25  \\ \hline 
                2A6&20633.18&44.10&198.36&0.17&14.90&0.13&0.49&0.01&121.79&1.04  \\ \hline 
                2A7&20470.14&43.72&193.59&0.16&15.80&0.09&0.46&0.01&132.32&0.77  \\ \hline 
                2A8&20658.61&44.82&192.15&0.17&15.76&0.13&0.47&0.01&132.95&1.07  \\ \hline 
                \end{tabular}
        \end{adjustbox}

\end{table}

\begin{figure}
        \begin{multicols}{2}
                \includegraphics[width=0.5\textwidth]{scurve/Generowana_czestotliwosc_fit_fe.jpg} \par
                \includegraphics[width=0.5\textwidth]{scurve/srednia_fit_fe.jpg} \par       
        \end{multicols} \hfill
        \begin{multicols}{2}
                \includegraphics[width=0.5\textwidth]{scurve/RCS_fit_fe.jpg} \par
                \includegraphics[width=0.5\textwidth]{scurve/ENC_fit_fe.jpg} \par
        \end{multicols}
        \caption{Wartości współczynników dopasowanych do krzywych s \ref{s curve fe}}
        \label{wyk wsp fe}
\end{figure}